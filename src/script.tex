\documentclass{article}

\usepackage[italian]{babel} % set the language to italian
\usepackage{graphicx} % used to insert pictures
\usepackage{mathtools} % imports some math tools
\usepackage{indentfirst} % adds the spacing at the beginning of every paragraph
\usepackage[margin = 1.0in]{geometry} % changes the size of the margins
\usepackage{pgfplots} % used to plot graphs
\usepackage{hyperref} % used for hyperlinks

\setlength{\parindent}{3em} % changes the size of the paragraph spacing

% check documentation for the lines below
\setlength{\voffset}{-0.5in}
\setlength{\footskip}{0.4in}
\setlength{\textheight}{690pt}
\pgfplotsset{width = 10cm, compat = 1.9} 

\definecolor{light_green}{RGB}{20, 150, 63}
\definecolor{light_purple}{RGB}{131, 118, 227}

\begin{document}

\title{\textbf{La composizione relativistica delle velocità}}
\author{\textit{Alessio Bandiera}}
\date{} % empty date, so no date will show up in the document

\maketitle

\section{Problema di partenza}
\null\par
Consideriamo un punto materiale all'interno di un sistema
di riferimento inerziale \(S\), che sia in movimento con
velocità \(u\). Ora consideriamo un secondo sistema di
riferimento inerziale \(S'\), in movimento con velocità
\(v\) rispetto ad \(S\), ed in modo tale che  \(v\) sia parallela
agli assi sovrapposti \(x\) ed \(x'\), che \(S\) ed \(S'\) siano
paralleli ed equiversi, e che all'istante \(t=t'=0\) s, \(S\)
ed \(S'\) si sovrappongano; rispetto ad \(S'\), il punto
materiale in \(S\) si muove con velocità \(u'\). \`E possibile
trovare una legge che sia in grado di mettere in relazione
le due velocità \(u\) ed \(u'\)?

\section{Soluzione trovata da Galileo}
La soluzione è stata data per la prima volta da Galileo Galilei,
intorno agli inizi del 1600, il quale era riuscito a formulare
delle trasformazioni che permettessero di mettere in relazione
le 4 dimensioni (\textit{3 spaziali e quella temporale}), di due
sistemi di riferimento inerziali, nella condizione in cui uno
dei due fosse in moto relativamente rispetto all'altro.

\begin{equation}
    \left\{
        \begin{aligned}
        x' &= x - vt \\
        y' &= y \\
        z' &= z \\
        t' &= t
        \end{aligned}
    \right.
\end{equation}

Tali
trasformazioni prendono il nome di ``\textbf{\textit{Trasformazioni di Galileo}}",
dalle quali è possibile ricavare la legge che mette in relazione
le due velocità del punto materiale, \(u\) e \(u'\), rispetto
ad \(S\) e \(S'\):

\begin{equation}
    \label{velocità di Galileo}
    u' = u - v
\end{equation}

\begin{figure}[htbp] % `[htpb]` puts the picture under the text above it (and not in another page)
    \label{galileo}
    \centerline{\includegraphics[scale=0.15]{galileo.jpg}}
    \caption{Ritratto di Galileo Galilei}
\end{figure}

\section{La relatività ristretta}
Nel 1905, tre secoli dopo Galileo, Albert Einstein sviluppò la
teoria che rivoluzionò per sempre la fisica:
\textbf{la teoria della relatività}. Einstein mise in discussione
una grandezza fisica che nessuno, prima di lui, aveva mai pensato
di analizzare da un punto di vista più relativo: il tempo. Infatti,
all'interno della teoria della relatività (\textit{in particolare quella
ristretta}), Einstein spiega che il tempo non è una grandezza
assoluta, e ciò deriva dai due postulati sui quali si fonda
la sua teoria:

\begin{itemize}
    \item{\textbf{Le leggi della meccanica, dell'elettromagnetismo e
    dell'ottica sono le stesse in tutti i sistemi di riferimento inerziali}}
    \item{\textbf{La luce si propaga nel vuoto a velocità costante \(c\),
    indipendentemente dallo stato di moto della sorgente o dell'osservatore}}
\end{itemize}

\section{Il nuovo problema}
In quanto la luce, secondo la teoria della relatività, viaggia alla
stessa velocità in tutti i sistemi di riferimento inerziali,
le trasformazioni che Galileo aveva dedotto tre secoli prima non si
dimostravano più valide per velocità prossime a quelle della luce:
questo, in quanto le trasformazioni galileiane non ammettono invarianti,
in disaccordo con la teoria della relatività, che prevede \(c\) come
invariante in ogni sistema di riferimento.

\begin{equation}
    \label{velocità della luce}
    c = 299.792 \textrm{\ km/s}
\end{equation}

Infatti, ad esempio, prendendo valori come \(u = \frac{2}{3}c\) e \(v = - \frac{2}{3}c\),
(\textit{ovvero, il sistema di riferimento \(S'\) si muoverebbe in verso opposto rispetto
al sistema \(S\)}) allora, applicando la legge derivata dalle trasformazioni
di Galileo, otterremmo

\begin{equation}
    u' = u - v = \frac{2}{3}c - (- \frac{2}{3}c) = \frac{4}{3}c
\end{equation}

Ma questo valore non può essere ritenuto valido, in quanto nessuna
velocità può superare quella della luce; dunque, risulta ovvio che le
trasformazioni di Galileo debbano essere necessariamente modificate.

\begin{figure}[htbp]
    \label{einstein}
    \centerline{\includegraphics[scale=0.2]{einstein.jpg}}
    \caption{Albert Einstein}
\end{figure}

\section{Le trasformazioni di Lorentz}
Grazie ai contributi dati, inizialmente da Larmor nel 1887, successivamente
da Poincarè nel 1905, alla relatività ristretta, vennero definite le
cosiddette ``\textbf{\textit{Trasformazioni di Lorentz}}". Tale nome venne
scelto da Larmor stesso, in quanto queste trasformazioni sono caratterizzate
dalla presenza del cosiddetto ``\textit{fattore Lorentziano}", indicato con
la lettera greca \(\gamma\), ed è pari a

\begin{equation}
    \label{gamma}
    \gamma = \frac{1}{\sqrt{1 - \frac{v^2}{c^2}}} 
\end{equation}

Le trasformazioni di Lorentz sono trasformazioni lineari di coordinate
che permettono di descrivere come variano le misure del tempo e dello spazio,
tra due sistemi di riferimento inerziali, riuscendo a tenere conto
dell'invarianza della velocità della luce:

\begin{equation}
    \left\{
        \begin{aligned}
        x' &= \gamma\ (x - vt) \\
        y' &= y \\
        z' &= z \\
        t' &= \gamma \left(t - \frac{v}{c^2} x\right)
        \end{aligned}
    \right.
\end{equation}

\section{La composizione relativistica delle velocità}
Mediante le trasformazioni di Lorentz, è possibile risolvere il
problema che caratterizzava le trasformazioni galileiane, e trovare
una formula che possa esprimere la relazione tra \(u\) e \(u'\),
che tenga in considerazione la velocità della luce:

\begin{equation}
    u' = \frac{u - v}{1 - \frac{uv}{c^2}}\ \ \ \ \ \ \ \ \ \ \ \ \ \ \ \ \ u = \frac{u' + v}{1 + \frac{u'v}{c^2}} % this is terrible, i know, i'm sorry, i'm lazy
\end{equation}

\subsection{Dimostrazione della formula}
A partire dall'uguaglianza

\begin{equation}
    \frac{\Delta x'}{\Delta t'} = \frac{x_2' - x_1'}{t_1' - t_1'}
\end{equation}

riscriviamo il secondo membro mediante le trasformazioni di Lorentz

\begin{equation}
    \left\{
        \begin{aligned}
        x' &= \gamma\ (x - vt) \\
        t' &= \gamma \left(t - \frac{v}{c^2} x\right)
        \end{aligned}
    \right.
    \Rightarrow
    \frac{\Delta x'}{\Delta t'} = \frac{\gamma(x_2 - vt_2) - \gamma(x_1 - vt_1)}{\gamma(t_2 - \frac{v}{c^2} x_2) - \gamma(t_1 - \frac{v}{c^2} x_1)}
\end{equation}

e, successivamente, semplificando \(\gamma\) e sciogliendo le parentesi, otterremo

\begin{equation}
    u' = \frac{\Delta x'}{\Delta t'} = \frac{x_2 - vt_2 - x_1 + vt_1}{t_2 - \frac{v}{c^2} x_2 - t_1 + \frac{v}{c^2} x_1} = \frac{(x_2 - x_1) - v(t_2 - t_1)}{(t_2 - t_1) - \frac{v}{c^2}(x_2 - x_1)}
\end{equation}

ma sapendo che \(x = ut\), e quindi \(\Delta x = u \Delta t\), allora

\begin{equation}
    u' = \frac{u(t_2 - t_1) - v(t_2 - t_1)}{(t_2 - t_1)-\frac{uv}{c^2}(t_2 - t_1)}
\end{equation}

ed infine

\begin{equation}
    u' = \frac{(t_2 - t_1)(u - v)}{(t_2 - t_1)\left(1 - \frac{uv}{c^2}\right)} = \frac{u - v}{1 - \frac{uv}{c^2}}
\end{equation}

\subsection{Invarianza di \textit{c}}
\`{E} possibile dimostrare che tale formula tiene conto dell'invarianza della velocità
della luce, semplicemente ponendo \(u = c\), e svolgendo i calcoli, si otterrà

\begin{equation}
    u' = \frac{c - v}{1 - \frac{cv}{c^2}} = \frac{c - v}{1 - \frac{v}{c}} = \frac{c - v}{\frac{c - v}{c}} = \frac{c - v}{c - v}c = c
\end{equation}

e dunque, per \(u = c\), \(u'\) sarà pari a \(c\) indipendentemente dalla velocità \(v\).

\subsection{Velocità molto piccole rispetto a \textit{c}}
Le trasformazioni di Galileo erano state ritenute valide fino alla
teoria della relatività, poiché è possibile dimostrare che per
velocità \(u\) ed \(u'\) di molto inferiori rispetto a \(c\),
le trasformazioni di Lorentz possono essere approssimate a quelle
di Galileo. Infatti, le trasformazioni di Galileo risultano essere un caso
particolare delle trasformazioni di Lorentz, in quanto ponendo
\(u << c\) e \(v << c\), allora \(\frac{uv}{c^2} \approx 0\), e dunque

\begin{equation}
    u' \approx \frac{u - v}{1 - 0} \approx u - v
\end{equation}

ovvero, approssimativamente la formula \((2)\).

\section{Analisi matematica della composzione relativistica delle velocità}
Per comodità, studieremo la funzione della composizione relativistica
delle velocità, nella forma

\begin{equation}
    u = \frac{u' + v}{1 + \frac{u'v}{c^2}}
\end{equation}

Inoltre, a partire da questa sezione in poi, la funzione
verrà analizzata da un punto di vista puramente matematico,
in quanto è bene ricordare che \(-c < u' < c\), ma diremo
che il dominio della funzione coincide con l'asse reale
privato del punto in cui il denominatore si annulla,
per semplicità di intenti (\textit{la condizione
\(-c < v < c\) verrà comunque tenuta in considerazione}).

\subsection{Funzione omografica}
La cosiddetta \textit{funzione omografica}, è una funzione del tipo

\begin{equation}
    y = \frac{ax + b}{cx + d}\ \ \ (c \neq 0)
\end{equation}

Ad esempio, il grafico di \(y = \frac{1}{x}\), un caso di funzione
omografica molto semplice, è il seguente:

\begin{center}
    \begin{tikzpicture}
        \begin{axis}[
            axis lines = center,
            thick,
            xlabel = {$x$},
            ylabel = {$y$},
        ]
            \addplot [
                domain = -10:-0.1, 
                samples = 100, 
                color = teal,
            ] {1 / x};

            \addplot [
                domain = 0.1:10,
                samples = 100, 
                color = teal,
            ] {1 / x};

            \legend{$\frac{1}{x}$};
        \end{axis}
    \end{tikzpicture}
\end{center}

La funzione \(u(u')\) risulta dunque essere proprio una
funzione omografica, ed in particolare

\begin{equation}
    a = 1\ \ \ \ \ b = v\ \ \ \ \ c = \frac{v}{c^2}\ \ \ \ \ d = 1
\end{equation}

\subsection{Studio della funzione}
Partiamo dal classificare la funzione: essa è una \textit{funzione
algebrica razionale fratta}, e non è definita per tutto l'asse reale,
infatti il dominio è

\begin{equation}
    \mathrm{D}_u = \forall u'\ |\ 1 + \frac{u'v}{c^2} \neq 0
\end{equation}

e dunque

\begin{equation}
    \mathrm{D}_u = \forall u'\ |\ u' \neq - \frac{c^2}{v}
\end{equation}

Le intersezioni della curva con gli assi possono essere calcolate
risolvendo due semplici sistemi:

\begin{equation}
    \left\{
        \begin{aligned}
        u' &= 0 \\
        u &= \frac{0 + v}{1 + 0}
        \end{aligned}
    \right.
    \ \ \
    \left\{
        \begin{aligned}
        u &= 0 \\
        0 &= \frac{u' + v}{1 + \frac{u'v}{c^2}}
        \end{aligned}
    \right.
    \Rightarrow
    \left\{
        \begin{aligned}
        u' &= 0 \\
        u &= v
        \end{aligned}
    \right.
    \ \ \
    \left\{
        \begin{aligned}
        u &= 0 \\
        u' &= -v
        \end{aligned}
    \right.
\end{equation}

e dunque i punti di intersezione con gli assi sono

\begin{equation}
    (0;\ v)\ \ \ (-v;\ 0)
\end{equation}

Successivamente, andiamo a studiare gli intervalli di \(u'\)
per i quali la funzione risulta essere positiva, negativa e nulla. Per
far questo, basterà imporre \(u > 0\), e dunque

\begin{equation}
    \frac{u' + v}{1 + \frac{u'v}{c^2}} > 0
\end{equation}

Studiando il segno della funzione, otterremo che il numeratore
è maggiore per

\begin{equation}
    u' + v > 0 \Rightarrow u ' > -v
\end{equation}

mentre il denominatore dipende dal segno di \(v\):

\begin{align*}
    v > 0:\ 1 + \frac{u'v}{c^2} > 0\ \Rightarrow\ u'v > -c^2\ \Rightarrow\ u' > - \frac{c^2}{v} \\
    v < 0:\ 1 + \frac{u'v}{c^2} > 0\ \Rightarrow\ u'v > -c^2\ \Rightarrow\ u' < - \frac{c^2}{v}
\end{align*}

facendo attenzione ad invertire il segno della disequazione
nell'ultimo passaggio, poiché stiamo dividendo per \(v < 0\).
Quindi, complessivamente, lo studio della funzione mostra che

% these abominations happen when you run out of time
\begin{align*}
    \ \ \ \ \ u > 0 &:\ u' < - \frac{c^2}{v} \vee u' > -v\ \ \ \ \ \ \ \ \ \ \ \ \ \ \ \ \ \ u > 0 :\ -v < u' < - \frac{c^2}{v} \\
    v > 0:\ \ u = 0 &:\ u' = -v\ \ \ \ \ \ \ \ \ \ \ \ \ \ \ \ \ \ \ \ \ \ v < 0:\ \ u = 0 :\ u' = -v \\
    \ \ \ \ \ u < 0 &:\ - \frac{c^2}{v} < u' < -v\ \ \ \ \ \ \ \ \ \ \ \ \ \ \ \ \ \ \ \ \ \ \ \ u < 0 :\ u' < -v \vee u' > - \frac{c^2}{v}
\end{align*}

in quanto, se \(v < 0\), allora \(-v < - \frac{c^2}{v}\)
(\textit{poiché \(v\) deve essere minore di \(c\)}).\hfill\break

\textit{Per quanto detto nel paragrafo \(7\), bisogna comunque
tenere a mente che, da un punto di vista fisico, la disequazione
\(u'v > -c^2\) è sempre verificata, poichè \(u'\) e \(v\)
saranno sempre maggiori di \(-c\), e dunque il primo membro, anche
nel caso in cui \(u'\) e \(v\) fossero discordi, non
sarà mai pari o inferiore a \(-c^2\).}\hfill\break

Procedendo con lo studio della funzione, andiamo alla ricerca
di eventuali limiti verticali, facendo tendere la variabile
indipendente al valore che non fa parte del dominio della
funzione (19), quindi calcolando

\begin{equation}
    \lim_{u' \rightarrow - \frac{c^2}{v}}{\frac{u' + v}{1 + \frac{u'v}{c^2}}}
\end{equation}

che è pari a

\begin{equation}
    \frac{- \frac{c^2}{v} + v}{1 - \frac{c^2v}{vc^2}} = \frac{\frac{-c^2 + v^2}{v}}{1 - 1} = \frac{v^2 - c^2}{0}
\end{equation}

Il numeratore di tale frazione è sempre negativo,
in quanto \(v < c\ \Rightarrow\ v^2 < c^2\ \Rightarrow\ v^2 - c^2 < 0\),
dunque

\begin{align*}
    \lim_{u' \rightarrow - \frac{c^2}{v}^-}{u} = + \infty \\
    \lim_{u' \rightarrow - \frac{c^2}{v}^+}{u} = - \infty
\end{align*}

quindi \(u' = - \frac{c^2}{v}\) risulta essere un asintoto verticale. Invece,
per quanto riguarda gli eventuali asintoti orizzontali, calcoliamo

\begin{equation}
    \lim_{u' \rightarrow \pm \infty}{\frac{u' + v}{1 + \frac{u'v}{c^2}}}
\end{equation}

e per calcolarlo, è necessario mettere in evidenza \(u'\) sia al numeratore
che al denominatore, e dunque possiamo riscrivere il limite come

\begin{equation}
    \lim_{u' \rightarrow \pm \infty}{\frac{u'(1 + \frac{v}{u'})}{u'(\frac{1}{u'} + \frac{v}{c^2})}}
\end{equation}

e semplificando \(u'\), possiamo snellire ulteriormente il limite,
in quanto se

\begin{equation}
    u' \rightarrow \pm \infty\ \Rightarrow\
    \begin{aligned}
        \frac{v}{u'} \rightarrow 0 \\
        \frac{1}{u'} \rightarrow 0
    \end{aligned}
\end{equation}

dunque

\begin{equation}
    \frac{(1 + 0)}{(0 + \frac{v}{c^2})} = \frac{1}{\frac{v}{c^2}} = \frac{c^2}{v}
\end{equation}

quindi \(u = \frac{c^2}{v}\) è un asintoto orizzontale. In seguito,
andremo a calcolare la derivata prima della funzione, per cercare
eventuali punti di non derivabilità, e punti di minimo e/o punti di massimo.
La derivata prima di \(u\) è possibile calcolarla ricordando che

\begin{equation}
    \frac{d}{dx} \frac{f(x)}{g(x)} = \frac{f'(x) g(x) - f(x) g'(x)}{g(x)^2}
\end{equation}

e quindi, la derivata di \(u(u')\) sarà

\begin{equation}
    \frac{d}{du'} u = \frac{(1 + \frac{u'v}{c^2}) - (u' + v) \frac{v}{c^2}}{\left(1 + \frac{u'v}{c^2}\right)^2} = \frac{1 + \frac{u'v}{c^2} - \frac{u'v}{c^2} - \frac{v^2}{c^2}}{\left(1 + \frac{u'v}{c^2}\right)^2} = \frac{1 - \frac{v^2}{c^2}}{\left(1 + \frac{u'v}{c^2}\right)^2}
\end{equation}

o, alternativamente, poiché \(\beta = \frac{v}{c}\), possiamo
riscrivere la derivata come

\begin{equation}
    \frac{d}{du'} u = \frac{1 - \beta^2}{\left(1 + \frac{u'v}{c^2}\right)^2}
\end{equation}

Ora, proseguiamo studiando il segno della derivata prima, e dunque

\begin{equation}
    1 - \beta^2 > 0\ \Rightarrow\ \beta^2 < 1\ \Rightarrow\ -1 < \beta < 1
\end{equation}

e poiché \(\beta = \frac{v}{c}\), allora

\begin{equation}
    -1 < \frac{v}{c} < 1\ \Rightarrow\ -c < v < c
\end{equation}

e ciò è sempre vero per il secondo assioma della relatività
ristretta, di conseguenza il numeratore è sempre positivo. Invece,
per quanto riguarda il numeratore, ponendolo maggiore di
0 otterremo

\begin{equation}
    \left(1 + \frac{u'v}{c^2}\right)^2 > 0
\end{equation}

ed essendo un quadrato, è verificato per

\begin{equation}
    \forall u'\ |\ 1 + \frac{u'v}{c^2} \neq 0
\end{equation}

ovvero

\begin{equation}
    \forall u'\ |\ u' \neq - \frac{c^2}{v}
\end{equation}

Tale punto non costituisce un punto di non derivabilità
in quanto il dominio della funzione è (19), e dunque
il punto \(u' = - \frac{c^2}{v}\) non fa parte del
dominio di \(u\). Di conseguenza, il denominatore è
anch'esso sempre positivo, dunque complessivamente
la funzione è sempre positiva, quindi si troverà nel
I e nel II quadrante, e presenta un asintoto verticale
che coincide con l'asintoto verticale della funzione
di partenza.\hfill\break

Dunque, tracciando il grafico della derivata prima,
otterremo il seguente (\textit{nel grafico riportato
sono stati utilizzati valori di \(c\) e di \(v\)
non realistici, in quanto sono stati posti \(c = 3\) e \(v = 2\),
ma esclusivamente per ragioni di comodità nel tracciare
il grafico, l'andamento di questo risulta lo stesso con i
valori reali, in quanto risulta solamente rappresentato
in scala di un fattore pari a circa \(10^8\)}):

% in this example, v = 2 and c = 3
\begin{center}
    \begin{tikzpicture}
        \begin{axis}[
            axis lines = center,
            thick,
            xlabel = {$u'$},
            ylabel = {$u$},
            legend pos = outer north east,
            legend style={/tikz/every even column/.append style = {row sep=0.5cm}},
        ]
            \addplot [
                domain = -6:-9 / 2 - 0.1, 
                samples = 100, 
                color = light_green,
            ] {(1 - 2^2 / 9) / (1 + 2 * x / 9)^2};

            \addplot +[
                mark = none,
                color = light_purple,
            ] coordinates {(-9 / 2, 0) (-9 / 2, 1130)};

            \addplot [
                domain = -9 / 2 + 0.1:1,
                samples = 100, 
                color = light_green,
            ] {(1 - 2^2 / 9) / (1 + 2 * x / 9)^2};

            \legend{$\frac{1 - \beta^2}{\left(1 + \frac{u'v}{c^2}\right)^2}$, $- \frac{c^2}{v}$};
        \end{axis}
    \end{tikzpicture}
\end{center}

L'ultimo elemento che bisogna analizzare per completare
lo studio della funzione è la derivata seconda. Ma prima di
calcolarla, riscriviamo la funzione utilizzando la formula (5)
che definisce il fattore Lorentziano:

\begin{equation}
    \begin{aligned}
        \frac{d}{du'} u = \frac{1 - \beta^2}{\left(1 + \frac{u'v}{c^2}\right)^2} \\
        \gamma = \frac{1}{\sqrt{1 - \beta^2}}\ \Rightarrow\ \gamma^{-2} = 1 - \beta^2
    \end{aligned}
    \Rightarrow\ \frac{d}{du'} u = \frac{1}{\gamma^2\left(1 + \frac{u'v}{c^2}\right)^2} = \frac{1}{\left[\gamma \left(1 + \frac{u'v}{c^2}\right)\right]^2}
\end{equation}

Utilizzando la formula (30), possiamo calcolare la derivata
seconda della funzione in esame, ottenendo

\begin{equation}
    \frac{d^2}{du'^2} u = \frac{-2\left[\gamma \left(1 + \frac{u'v}{c^2}\right)\right] \gamma \frac{v}{c^2}}{\left[\gamma \left(1 + \frac{u'v}{c^2}\right)\right]^4} = \frac{-2v}{c^2\gamma^2 \left(1 + \frac{u'v}{c^2}\right)^3}
\end{equation}

Studiandone il segno, poniamo la funzione maggiore di 0

\begin{equation}
    \frac{-2v}{c^2\gamma^2 \left(1 + \frac{u'v}{c^2}\right)^3} > 0
\end{equation}

e partendo dal denominatore,
osserviamo che la disequazione

\begin{equation}
    c^2\gamma^2 \left(1 + \frac{u'v}{c^2}\right)^3 > 0
\end{equation}

presenta il termine \(c^2 \gamma^2\),
il quale è sempre positivo, mentre il termine
\(\left(1 + \frac{u'v}{c^2}\right)^3\)
necessita di essere analizzato più approfonditamente.
Infatti, essendo un termine che vede \(v\)
come parametro, e potendo quest'ultimo assumere valore negativo,
per poter procedere con il calcolo è necessario
tenere in considerazione il suo segno. Infatti se \(v > 0\),
allora il numeratore \(-2v\) sarà complessivamente negativo, e
di conseguenza per essere positiva la frazione, il
denominatore deve assumere segno negativo, e dunque

\begin{equation}
    \left(1 + \frac{u'v}{c^2}\right)^3 < 0\ \Rightarrow\ 1 + \frac{u'v}{c^2} < 0\ \Rightarrow\ \frac{u'v}{c^2} < -1\ \Rightarrow\ u' < - \frac{c^2}{v}
\end{equation}

e in particolare, nell'ultimo passaggio non è
necessario invertire il segno della disequazione in
quanto \(v\) è positivo in ipotesi. Differentemente,
se poniamo \(v < 0\), allora il numeratore sarà
complessivamente positivo, e dunque per assumere
segno positivo la frazione, il denominatore dovrà
essere necessariamente positivo, quindi

\begin{equation}
    \left(1 + \frac{u'v}{c^2}\right)^3 > 0\ \Rightarrow\ 1 + \frac{u'v}{c^2} > 0\ \Rightarrow\ \frac{u'v}{c^2} > -1\ \Rightarrow\ u' < - \frac{c^2}{v}
\end{equation}

e, nell'ultimo passaggio, bisogna fare attenzione
ad invertire il segno della disequazione, in quanto
\(v < 0\) in ipotesi. Abbiamo dunque trovato un risultato
molto interessante: \textit{la concavità della funzione
non dipende dalla velocita \(v\) del secondo sistema di
riferimento}. Infatti, in entrambi i casi, la funzione
avrà la concavità verso l'alto per \(u' < - \frac{c^2}{v}\),
mentre avrà la concavità verso il basso per \(u' > - \frac{c^2}{v}\).

\section{Grafico della funzione}
Disponendo ora di tutti gli elementi necessari,
possiamo finalmente tracciare il grafico di \(u(u')\):

% in this example, v = 2 and c = 3
\begin{center}
    \begin{tikzpicture}
        \begin{axis}[
            axis lines = center,
            thick,
            xlabel = {$u'$},
            ylabel = {$u$},
            legend pos = outer north east,
            legend style={/tikz/every even column/.append style = {row sep=0.5cm}},
        ]
            \addplot [
                domain = -10:-9 / 2 - 0.5, 
                samples = 100, 
                color = light_green,
            ] {(x + 2) / (1 + x * 2 / 9)};

            \addplot +[
                mark = none,
                color = orange,
                dashed,
            ] coordinates {(-9 / 2, -18) (-9 / 2, 27)};

            \addplot +[
                mark = none,
                color = orange,
                dashed,
            ] coordinates {(-10, 9 / 2) (10, 9 / 2)};

            \addplot [
                domain = -9 / 2 + 0.5:10,
                samples = 100, 
                color = light_green,
            ] {(x + 2) / (1 + x * 2 / 9)};

            \legend{
                $\frac{u' + v}{1 + \frac{u'v}{c^2}}$,
                $- \frac{c^2}{v}$,
                $\frac{c^2}{v}$
            };
        \end{axis}
    \end{tikzpicture}
\end{center}

(\textit{nel grafico rappresentato valgono le stesse
considerazioni che sono state utilizzate per
tracciare il grafico della formula \((32)\)})\hfill\break

Tale grafico è stato rappresentato per un valore di
\(v\) positivo, poiché quando \(v < 0\), il valore
\(- \frac{c^2}{v}\) sarà positivo e non più negativo,
ed in maniera analoga, \(\frac{c^2}{v}\) sarà negativo;
di conseguenza, gli asintoti invertiranno il loro segno,
mentre la concavità, per quanto mostrato precedentemente,
non cambierà. Di seguito è riportato un grafico con
\(v < 0\), più precisamente il valore opposto al grafico
precedente (\textit{valgono le stesse considerazioni del
grafico precedente}):

\begin{center}
    \begin{tikzpicture}
        \begin{axis}[
            axis lines = center,
            thick,
            xlabel = {$u'$},
            ylabel = {$u$},
            legend pos = outer north east,
            legend style={/tikz/every even column/.append style = {row sep=0.1cm}},
        ]
            \addplot [
                domain = -10:9 / 2 - 0.5, 
                samples = 100, 
                color = light_green,
            ] {(x - 2) / (1 + x * (-2) / 9)};
            
            \addplot [color = white] {x - 10}; % i know, this solution is terrible, but i wasn't able to change the size of the legend box

            \addplot [
                domain = 9 / 2 + 0.5:10,
                samples = 100, 
                color = light_green,
            ] {(x - 2) / (1 + x * (-2) / 9)};

            \legend{$\frac{u' + v}{1 + \frac{u'v}{c^2}}$, $ $};
        \end{axis}
    \end{tikzpicture}
\end{center}

\section{Postfazione}
Questo elaborato, assieme alla presentazione visiva, è il
risultato di numerose ore di lavoro, ed è stato realizzato
utilizzando: \textbf{Manim}, per la parte delle animazioni
dell'elaborato; \textbf{LaTeX}, per la realizzazione di questo
PDF; \textbf{Visual Studio Code}, piattaforma sulla quale
è stato scritto tutto il codice dell'elaborato; \textbf{Python},
il linguaggio con cui è scritto Manim; \textbf{Desmos}, per la
realizzazione di grafici dinamici; \textbf{Sony Vegas Pro},
con il quale sono state rielaborate le animazioni prodotte
tramite Manim per adattarle ai tempi della presentazione;
\textbf{PowerPoint}, contenente le varie animazioni della
presentazione; infine, \textbf{Word}, utilizzato per scrivere
la bozza iniziale del contenuto da trattare.\hfill\break

L'elaborato è interamente \textit{open source},
ed è disponibile su \textbf{GitHub} al seguente link, all'interno
del quale è possibile trovare tutto il codice ed il materiale
utilizzato per realizzarlo:

\url{https://github.com/ph04/relativistic-addition}\hfill\break

Si ringrazia la commissione per l'attenzione.

\pagebreak

\tableofcontents

\end{document}
