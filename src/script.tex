\documentclass{article}

\usepackage[italian]{babel} % set the language to italian
\usepackage{graphicx} % used to insert pictures
\usepackage{mathtools} % imports some math tools
\usepackage{indentfirst} % adds the spacing at the beginning of every paragraph
\usepackage[margin = 1.0in]{geometry} % changes the size of the margins
\usepackage{pgfplots} % used to plot graphs
\usepackage{xcolor} % used to import more color names

\setlength{\parindent}{3em} % changes the size of the paragraph spacing

% check documentation for the lines below
\setlength{\voffset}{-0.5in}
\setlength{\footskip}{0.4in}
\setlength{\textheight}{690pt}
\pgfplotsset{width = 10cm,compat = 1.9} 

\begin{document}

\title{\textbf{La composizione relativistica delle velocità}}
\author{\textit{Alessio Bandiera}}
\date{} % empty date, so no date will show up in the document

\maketitle

\section{Problema di partenza}
\null\par
Consideriamo un punto materiale all'ìnterno di un sistema
di riferimento inerziale \(S\), che sia in movimento con
velocità \(u\). Ora consideriamo un secondo sistema di
riferimento inerziale \(S'\), in movimento con velocità
\(v\) rispetto ad \(S\), ed in modo tale che  \(v\) sia parallela
agli assi sovrapposti \(x\) ed \(x'\), che \(S\) ed \(S'\) siano
paralleli ed equiversi, e che all'istante \(t=t'=0\) s, \(S\)
ed \(S'\) si sovrappongano; rispetto ad \(S'\), il punto
materiale in \(S\) si muove di velocità \(u'\). \`E possibile
trovare una legge che sia in grado di mettere in relazione
le due velocità \(u\) ed \(u'\)?

\section{Soluzione trovata da Galileo}
La soluzione è stata data per la prima volta da Galieo Galilei,
intorno agli inizi del 1600, il quale era riuscito a formulare
delle trasformazioni che permettessero di mettere in relazione
le 4 dimensioni (\textit{3 spaziali e quella temporale}), di due
sistemi di riferimento inerziali, nella condizione in cui uno
dei due fosse in moto relativamente rispetto all'altro.

\begin{equation}
    \left\{
        \begin{aligned}
        x' &= x - vt \\
        y' &= y \\
        z' &= z \\
        t' &= t
        \end{aligned}
    \right.
\end{equation}

Tali
trasformazioni prendono il nome di "\textbf{\textit{Trasformazioni di Galileo}}",
dalle quali è possibile ricavare la legge che mette in relazione
le due velocità del punto materiale, \(u\) e \(u'\), rispetto
ad \(S\) e \(S'\):

\begin{equation}
    \label{velocità di Galileo}
    u' = u - v
\end{equation}

\begin{figure}[htbp] % `[htpb]` puts the picture under the text above it (and not in another page)
    \label{galileo}
    \centerline{\includegraphics[scale=0.15]{galileo.jpg}}
    \caption{Ritratto di Galileo Galilei}
\end{figure}

\section{La relatività ristretta}
Nel 1905, tre secoli dopo Galileo, Albert Einstein sviluppò la
teoria che rivoluzionò per sempre la fisica:
\textbf{la teoria della relatività}. Einstein mise in discussione
una grandezza fisica che nessuno, prima di lui, aveva mai pensato
di analizzare da un punto di vista più relativo: il tempo. Infatti,
all'interno della teoria della relatività (\textit{in particolare quella
ristretta}), Einstein spiega che il tempo non è una grandezza
assoluta, e ciò deriva dai due postulati sui quali si fonda
la sua teoria.

\begin{itemize}
    \item{\textbf{Le leggi della meccanica, dell'elettromagnetismo e
    dell'ottica sono le stesse in tutti i sistemi di riferimento inerziali}}
    \item{\textbf{La luce si propaga nel vuoto a velocità costante \(c\),
    indipendentemente dallo stato di moto della sorgente o dell'osservatore}}
\end{itemize}

\section{Il nuovo problema}
In quanto la luce, secondo la teoria della relatività, viaggia alla
stessa velocità in tutti i sistemi di riferimento inerziali,
le trasformazioni che Galileo aveva dedotto tre secoli prima non si
dimostravano più valide per velocità prossime a quelle della luce:
questo, in quanto le trasformazioni galileiane non ammettono invarianti,
in disaccordo con la teoria della relatività, che prevede \(c\) come
invariante in ogni sistema di riferimento.

\begin{equation}
    \label{velocità della luce}
    c = 299.792 \textrm{\ km/s}
\end{equation}

Infatti, ad esempio, prendendo valori come \(u = \frac{2}{3}c\) e \(v = - \frac{2}{3}c\),
(\textit{ovvero, il sistema di riferimento \(S'\) si muoverebbe in verso opposto rispetto
al sistema \(S\)}) allora, applicando la legge derivata dalle trasformazioni
di Galileo, otterremmo

\begin{equation}
    u' = u - v = \frac{2}{3}c - (- \frac{2}{3}c) = \frac{4}{3}c
\end{equation}

Ma questo valore non può essere ritenuto valido, in quanto nessuna
velocità può superare quella della luce; dunque, risulta ovvio che le
trasformazioni di Galileo debbano essere necessariamente modificate.

\begin{figure}[htbp]
    \label{einstein}
    \centerline{\includegraphics[scale=0.2]{einstein.jpg}}
    \caption{Albert Einstein}
\end{figure}

\section{Le trasformazioni di Lorentz}
Grazie ai contributi dati, inizialmente da Larmor nel 1887, successivamente
da Poincarè nel 1905, alla relatività ristretta, vennero definite le
cosiddette "\textbf{\textit{Trasformazioni di Lorentz}}". Tale nome venne
scelto da Larmor stesso, in quanto queste trasformazioni sono caratterizzate
dalla presenza del cosiddetto "\textit{fattore Lorentziano}", indicato con
la lettera greca \(\gamma\), ed è pari a

\begin{equation}
    \label{gamma}
    \gamma = \frac{1}{\sqrt{1 - \frac{v^2}{c^2}}} 
\end{equation}

Le trasformazioni di Lorentz sono trasformazioni lineari di coordinate
che permettono di descrivere come variano le misure del tempo e dello spazio,
tra due sistemi di riferimento inerziali, riuscendo a tenere conto
dell'invarianza della velocità della luce.

\begin{equation}
    \left\{
        \begin{aligned}
        x' &= \gamma\ (x - vt) \\
        y' &= y \\
        z' &= z \\
        t' &= \gamma \left(t - \frac{v}{c^2} x\right)
        \end{aligned}
    \right.
\end{equation}

\section{La composizione relativistica delle velocità}
Mediante le trasformazioni di Lorentz, è possibile risolvere il
problema che caratterizzava le trasformazioni galileiane, e trovare
una formula che possa esprimere la relazione tra \(u\) e \(u'\),
che tenga in considerazione la velocità della luce:

\begin{equation}
    u' = \frac{u - v}{1 - \frac{uv}{c^2}}\ \ \ \ \ \ \ \ \ \ \ \ \ \ \ \ \ u = \frac{u' + v}{1 + \frac{u'v}{c^2}} % this is terrible, i know, i'm sorry, i'm lazy
\end{equation}

\subsection{Dimostrazione della formula}
A partire dall'uguaglianza

\begin{equation}
    \frac{\Delta x'}{\Delta t'} = \frac{x_2' - x_1'}{t_1' - t_1'}
\end{equation}

riscriviamo il secondo membro mediante le trasformazioni di Lorentz

\begin{equation}
    \left\{
        \begin{aligned}
        x' &= \gamma\ (x - vt) \\
        t' &= \gamma \left(t - \frac{v}{c^2} x\right)
        \end{aligned}
    \right.
    \Rightarrow
    \frac{\Delta x'}{\Delta t'} = \frac{(x_2 - vt_2) - (x_1 - vt_1)}{(t_2 - \frac{v}{c^2} x_2) - (t_1 - \frac{v}{c^2} x_1)}
\end{equation}

e, successivamente, svolgendo i calcoli, otterremo

\begin{equation}
    u' = \frac{\Delta x'}{\Delta t'} = \frac{x_2 - vt_2 - x_1 + vt_1}{t_2 - \frac{v}{c^2} x_2 - t_1 + \frac{v}{c^2} x_1} = \frac{(x_2 - x_1) - v(t_2 - t_1)}{(t_2 - t_1) - \frac{v}{c^2}(x_2 - x_1)}
\end{equation}

ma sapendo che \(x = ut\), e quindi \(\Delta x = u \Delta t\), allora

\begin{equation}
    u' = \frac{u(t_2 - t_1) - v(t_2 - t_1)}{(t_2 - t_1)-\frac{uv}{c^2}(t_2 - t_1)}
\end{equation}

ed infine

\begin{equation}
    u' = \frac{(t_2 - t_1)(u - v)}{(t_2 - t_1)\left(1 - \frac{uv}{c^2}\right)} = \frac{u - v}{1 - \frac{uv}{c^2}}
\end{equation}

\subsection{Invarianza di \(c\)}
\`{E} possibile dimostrare che tale formula tiene conto dell'invarianza della velocità
della luce, semplicemente ponendo \(u = c\), e svolgendo i calcoli, si otterrà

\begin{equation}
    u' = \frac{c - v}{1 - \frac{cv}{c^2}} = \frac{c - v}{1 - \frac{v}{c}} = \frac{c - v}{\frac{c - v}{c}} = \frac{c - v}{c - v}c = c
\end{equation}

e dunque, per \(u = c\), \(u'\) sarà pari a \(c\) indipendentemente dalla velocità \(v\).

\subsection{Velocità molto piccole rispetto a \(c\)}
Le trasformazioni di Galileo erano state ritenute valide fino alla
teoria della relatività, poiché è possibile dimostrare che per
velocità \(u\) ed \(u'\) di molto inferiori rispetto a \(c\),
le trasformazioni di Lorentz possono essere approssimate a quelle
di Galileo. Infatti, le trasformazioni di Galileo risultano essere un caso
particolare delle trasformazioni di Lorentz, in quanto ponendo
\(u << c\) e \(v << c\), allora \(\frac{uv}{c^2} \approx 0\), e dunque

\begin{equation}
    u' \approx \frac{u - v}{1 - 0} \approx u - v
\end{equation}

ovvero, approssimativamente la formula \((2)\).

\section{Analisi matematica della composzione relativistica delle velocità}
Per comodità, studieremo la funzione della composizione relativistica
delle velocità, nella forma

\begin{equation}
    u = \frac{u' + v}{1 + \frac{u'v}{c^2}}
\end{equation}

\subsection{Funzione omografica}
% [FUNZIONE OMOGRAFICA]
La cosiddetta \textit{funzione omografica}, è una funzione del tipo

\begin{equation}
    y = \frac{ax + b}{cx + d}\ \ \ (c \neq 0)
\end{equation}

Ad esempio, il grafico di \(y = \frac{1}{x}\), un caso di funzione
omografica molto semplice, è

\begin{center}
\begin{tikzpicture}
    \begin{axis}[
        axis lines = center,
        xlabel = $x$,
        ylabel = {$y$},
    ]
        \addplot [
            domain = -10:-0.1, 
            samples = 100, 
            color = teal,
        ] {1 / x};
        \addplot [
            domain = 0.1:10,
            samples = 100, 
            color = teal,
        ] {1 / x};
        \addlegendentry{$\frac{1}{x}$};
    \end{axis}
\end{tikzpicture}
\end{center}

Inoltre, la funzione \(u(u')\) risulta proprio essere una
funzione omografica, ed in particolare

\begin{equation}
    a = 1\ \ \ \ \ b = v\ \ \ \ \ c = \frac{v}{c^2}\ \ \ \ \ d = 1
\end{equation}

\subsection{Studio della funzione}
% [STUDIO DELLA FUNZIONE]
Partiamo dal classificare la funzione: essa è una \textit{funzione
algebrica razionale fratta}, e non è definita per tutto l'asse reale,
infatti il dominio è

\begin{equation}
    D_u = \forall u'\ |\ 1 + \frac{u'v}{c^2} \neq 0
\end{equation}

e dunque

\begin{equation}
    D_u = \forall u'\ |\ u' \neq - \frac{c^2}{v}
\end{equation}

Le intersezioni della curva con gli assi possono essere calcolate
risolvendo due semplici sistemi:

\begin{equation}
    \left\{
        \begin{aligned}
        u' &= 0 \\
        u &= \frac{0 + v}{1 + 0}
        \end{aligned}
    \right.
    \ \ \
    \left\{
        \begin{aligned}
        u &= 0 \\
        0 &= \frac{u' + v}{1 + \frac{u'v}{c^2}}
        \end{aligned}
    \right.
    \Rightarrow
    \left\{
        \begin{aligned}
        u' &= 0 \\
        u &= v
        \end{aligned}
    \right.
    \ \ \
    \left\{
        \begin{aligned}
        u &= 0 \\
        u' &= -v
        \end{aligned}
    \right.
\end{equation}

e dunque i punti di intersezione con gli assi sono

\begin{equation}
    (0;\ v)\ \ \ (-v;\ 0)
\end{equation}


Successivamente, andiamo a studiare gli intervalli di \(u'\)
per i quali la funzione risulta essere positiva, negativa e nulla. Per
far questo, basterà imporre \(u > 0\), e dunque

\begin{equation}
    \frac{u' + v}{1 + \frac{u'v}{c^2}} > 0
\end{equation}

Facendo lo studio del segno della funzione, otterremo che il numeratore
è maggiore per

\begin{equation}
    u' + v > ' \Rightarrow u ' > -v
\end{equation}

mentre il denominatore è maggiore per

\begin{equation}
    1 + \frac{u'v}{c^2} > 0 \Rightarrow u' > - \frac{u'v}{c^2}
\end{equation}

Quindi, complessivamente, lo studio della funzione mostra che

% TODO: keep from here

\subsection{Analisi della derivata prima}
% [CALCOLO DELLA DERIVATA PRIMA]
% [STUDIO DEL SEGNO DELLA DERIVATA PRIMA]
Placeholder

\subsection{Analisi della derivata seconda}
% [CALCOLO DELLA DERIVATA SECONDA E STUDIO DEL SEGNO]
yo mamma's so overweight

\end{document}
